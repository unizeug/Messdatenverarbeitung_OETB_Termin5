\newcommand{\institut}{Institut f\"ur Energie und  Automatisiertungstechnik}
\newcommand{\fachgebiet}{Elektronische Mess- und Diagnosetechnik}
\newcommand{\veranstaltung}{Praktikum Messdatenverarbeitung}
\newcommand{\pdfautor}{\"Ozg\"u Dogan (326 048), Timo Lausen (325 411), Boris Henckell (325 779)}
\newcommand{\autor}{\"Ozg\"u Dogan (326 048)\\ Timo Lausen (325 411)\\ Boris Henckell (325 779)}
\newcommand{\pdftitle}{Praktikum Messdatenverarbeitung  Termin 3 und 4}
\newcommand{\prototitle}{Praktikum Messdatenverarbeitung \\ Termin 3 und 4}
\newcommand{\aufgabe}{}

\newcommand{\gruppe}{Gruppe: G1 Fr 08-10}
\newcommand{\betreuer}{Betreuer: J\"urgen Funk}

\input{../../packages/tu_header_8}
%\begin{document}

% \lstlistoflistings
\definecolor{darkgray}{rgb}{0.95,0.95,0.95}
\definecolor{darkolivegreen}{HTML}{01a801}
\definecolor{functionsBlue}{HTML}{32b9b9}
\definecolor{variableRed}{rgb}{1,0,0}
\definecolor{stringBrown}{HTML}{bc8e8e} % f geht nicht

\lstset{
        %\lstset{extendedchars=true} % Umlaute an der richtigen stelle und nicht am Anfang ausgeben
        %basicstyle=\footnotesize\ttfamily,
        basicstyle=\small,
        %
        inputencoding=utf8,
        %
        tabsize=4,
        showspaces=false,
        showtabs=false,
        showstringspaces=true, % no special string spaces
        %
        backgroundcolor=\color{darkgray}, % background
        stringstyle=\color{stringBrown}\fseries, % Strings
        keywordstyle=\color{functionsBlue}\bfseries, % keywords Blau
        identifierstyle=\color{variableRed}, % variablen
        commentstyle=\color{darkolivegreen}, %  comments
        %
        breaklines=true,
        %
        numbers=left,
        numberstyle=\tiny,
        stepnumber=1,
        numbersep=7pt,
        %
        frame=single,
        columns=flexible,
        %
        xleftmargin=-2cm,
        xrightmargin=-1.5cm,
        %
        language=Matlab
}


%---------------------------------------------------------------------
%---------------------------------------------------------------------
%---------------------------------------------------------------------

\TODO{Özgü: nochmal Korrekturlesen, TODOs entfernen, Vollständigkeit beachten}
\TODO{Boris: Bilder schöner einfügen, auf Gesamtform
achten, Vollständigkeit beachten}

\section{Vorbereitungsaufgaben}
\begin{quote}
    \subsection{Vorbereitungsaufgaben Termin 3}
    \begin{quote}
        \subsubsection{Sinusfunktion erzeugen und DFT erstellen}
        \begin{quote}
            Die erste Vorbereitungsaufgabe hatte das Ziel einen verschobenes Sinussignal zu erzeugen, die DFT davon zu
            betimmen und dieses zu plotten. Der dazugehärige Quelltext sinus2.m befindet sich im Anhang.
            \begin{figure}[H]
            \centering
                \includegraphics[scale=0.5, trim = 1cm 6cm 1.5cm 8cm, clip]{./Bilder/VerschobenerSinusAufgabe1}
                    \caption{Verschobener Sinus}
                    \label{fig:./Bilder/VerschobenerSinusAufgabe1}
            \end{figure}
        
        \end{quote}
        
    	\subsubsection{Angeschnittene Sinusfunktion}
        \begin{quote}
            In der zweiten Vorbereitungsaufgabe sollte eine Matlab-Funktion geschrieben werden die einen angeschnittene
            Sinusfunktion simuliuliert.\\
            Um diese Funktion zu erstellen haben wir eine for Schleife imlementiert, die den gesamten zeitvektor t
            durchläuft. Für jeden Durchlauf wird getestet, wo sich der jeweilige Zeitpunkt im Verhältniss zum halben
            sinussignal befindet. Anschließend wird in der if abfrage bestimmt, ob sich dieser Zeitpunkt relativ zur halben
            Periode des Sinussignals vor oder hinter dem Phasenanschnittswinkel befindet. Abhängig davon wird der
            dentsprechenden Stelle im Ausgabevektor der Wert $0$ oder der Wert den die Sinusfunktion ermittelt, übergeben.\\
            Wir haben die Graphen für folgende Phasenanschnittswinkel $ \alpha = 0, \frac{1}{8} \pi, \frac{1}{4}
            \pi, \frac{3}{8} \pi, \frac{1}{2} \pi,\frac{5}{8} \pi, \frac{3}{4} \pi, \frac{7}{8} \pi$ und $\pi$
            geplottet.\\
            Der dazugehärige Quelltext befindet sich im Anhang.
        \end{quote}
        
        \subsubsection{Effektivwert des Stroms im Zeitbereich}
        \begin{quote}
            Für den Effektivwert des Stroms im Zeitbereich ermitteln wir die Wurzel des Mittelwertes des Quadrats des
            Stromvektors.\\
            Der dazugehörige Quelltext befindet sich im Anhang.
        \end{quote}
        
        \subsubsection{Effektivwert des Stroms im Frequenzbereich}
        \begin{quote}
            Den Effektivwert des Stroms im Frequenzbereich ermitteln wir ähnlich wie den Effektivwert des Stroms im
            Zeitbereich. Das Parsevalschen-Theorem besagt, dass die Energien eines Signals im Zeitbereich gleich seiner
            Energie im Frequenzbereich ist.\cite{PasevalscheTheorem}\\
            \begin{equation*}
            	\begin{split}
            		\sum_{n=0}^{N-1} |x[n]|^2 = \frac{1}{N} \sum_{k=0}^{N-1} |X[k]|^2
            	\end{split}
            \end{equation*}
            
            Der dazugehörige Quelltext befindet sich im Anhang.
            \end{quote}
        \subsubsection{Ergebnisse Vorbereitungsaufgaben}
        \begin{quote}
                                    %4 Grafiken:
                \begin{center}
                \begin{tabular}{ll}
    
                \hspace{-11em}
                    \begin{minipage}{0.6\textwidth}
    
                        \begin{figure}[H]
                            \label{fig:}
                            \includegraphics[scale=0.5, trim = 1.5cm 7cm 1.5cm 8.5cm,
                            clip]{./Bilder/Phasenanschnitt08pi.pdf}
                            %FIXME [width=640px,
                             %height=474px]
                            \caption{Sinussignal mit Phasenanschnitt von $0$}
                        \end{figure}
    
                    \end{minipage}
                    \begin{minipage}{0.6\textwidth}
    
                        \begin{figure}[H]
                            \label{fig:}
                            \includegraphics[scale=0.5, trim = 1.5cm 7cm 1.5cm 8.5cm,
                            clip]{./Bilder/Phasenanschnitt18pi.pdf}
                            %FIXME [width=640px,
                             %height=474px]
                            \caption{Sinussignal mit Phasenanschnitt von $\frac{1}{8}/pi$}
                        \end{figure}
                    \vspace{-1.5em}
    
                    \end{minipage}
    
                \end{tabular}
                \end{center}
    
                            %4 Grafiken:
                \begin{center}
                \begin{tabular}{ll}
    
                \hspace{-11em}
                    \begin{minipage}{0.6\textwidth}
    
                        \begin{figure}[H]
                            \label{fig:}
                            \includegraphics[scale=0.5, trim = 1.5cm 7cm 1.5cm 8.5cm,
                            clip]{./Bilder/Phasenanschnitt28pi.pdf} %FIXME [width=640px, height=474px]
                            \caption{Sinussignal mit Phasenanschnitt von $\frac{1}{4}/pi$}
                        \end{figure}
    
                    \end{minipage}
                    \begin{minipage}{0.6\textwidth}
    
                         \begin{figure}[H]
                            \label{fig:}
                            \includegraphics[scale=0.5, trim = 1.5cm 7cm 1.5cm 8.5cm,
                            clip]{./Bilder/Phasenanschnitt38pi.pdf} %FIXME [width=640px, height=474px]
                            \caption{Sinussignal mit Phasenanschnitt von $\frac{3}{8}/pi$}
                        \end{figure}
                   \vspace{-1.5em}
    
                    \end{minipage}
    
                \end{tabular}
                \end{center}
    
                            %4 Grafiken:
                \begin{center}
                \begin{tabular}{ll}
    
                \hspace{-11em}
                    \begin{minipage}{0.6\textwidth}
    
                        \begin{figure}[H]
                            \label{fig:}
                            \includegraphics[scale=0.5, trim = 1.5cm 7cm 1.5cm 8.5cm,
                            clip]{./Bilder/Phasenanschnitt48pi.pdf} %FIXME [width=640px, height=474px]
                            \caption{Sinussignal mit Phasenanschnitt von $\frac{1}{2}/pi$}
                        \end{figure}
    
                    \end{minipage}
                    \begin{minipage}{0.6\textwidth}
    
                       \begin{figure}[H]
                            \label{fig:}
                            \includegraphics[scale=0.5, trim = 1.5cm 7cm 1.5cm 8.5cm,
                            clip]{./Bilder/Phasenanschnitt58pi.pdf} %FIXME [width=640px, height=474px]
                            \caption{Sinussignal mit Phasenanschnitt von $\frac{5}{8}/pi$}
                        \end{figure}
                     \vspace{-1.5em}
    
                    \end{minipage}
    
                \end{tabular}
                \end{center}
    
                   %4 Grafiken:
                \begin{center}
                \begin{tabular}{ll}
    
                \hspace{-11em}
                    \begin{minipage}{0.6\textwidth}
    
                        \begin{figure}[H]
                            \label{fig:}
                            \includegraphics[scale=0.5, trim = 1.5cm 7cm 1.5cm 8.5cm,
                            clip]{./Bilder/Phasenanschnitt68pi.pdf} %FIXME [width=640px, height=474px]
                            \caption{Sinussignal mit Phasenanschnitt von $\frac{3}{4}/pi$}
                        \end{figure}
    
                    \end{minipage}
                    \begin{minipage}{0.6\textwidth}
    
                       \begin{figure}[H]
                            \label{fig:}
                            \includegraphics[scale=0.5, trim = 1.5cm 7cm 1.5cm 8.5cm,
                            clip]{./Bilder/Phasenanschnitt78pi.pdf} %FIXME [width=640px, height=474px]
                            \caption{Sinussignal mit Phasenanschnitt von $\frac{7}{8}/pi$}
                        \end{figure}
                     \vspace{-1.5em}
    
                    \end{minipage}
    
                \end{tabular}
                \end{center}
                
                   %4 Grafiken:
                \begin{center}
                \begin{tabular}{ll}
    
                \hspace{-4em}
                    \begin{minipage}{0.6\textwidth}
    
                        \begin{figure}[H]
                            \label{fig:}
                            \includegraphics[scale=0.5, trim = 1.5cm 7cm 1.5cm 8.5cm,
                            clip]{./Bilder/Phasenanschnitt88pi.pdf} %FIXME [width=640px, height=474px]
                            \caption{Sinussignal mit Phasenanschnitt von $/pi$}
                        \end{figure}
    
                    \end{minipage}
    
                \end{tabular}
                \end{center}
                 \begin{center}
                     \begin{tabular}{|c|c|c|}
                                 
                       \hline
                       $\alpha $ & $i_{eff}$ (Zeitbereich) & $i_{eff}$ Frequenzbereich\\ \hline
                       $0$ & 3,5326 & 3,5326 \\ \hline
                       $\frac{1}{8} \pi$ & 3,5105 & 3,5105 \\ \hline
                       $\frac{1}{4} \pi$ & 3,3744 & 3,3744 \\ \hline
                       $\frac{3}{8} \pi$ & 3,0338 & 3,0338 \\ \hline
                       $\frac{1}{2} \pi$ & 2,5145 & 2,5145 \\ \hline
                       $\frac{5}{8} \pi$ & 1,8097 & 1,8097 \\ \hline
                       $\frac{3}{4} \pi$ & 1,0748 & 1,0748 \\ \hline
                       $\frac{7}{8} \pi$ & 0,3940 & 0,3940 \\ \hline
                       $ \pi$ & 0 & 0 \\ \hline
                             
               
                     \end{tabular}
                 \end{center}        
        \end{quote}
    \end{quote}
    \subsection{Vorbereitungsaufgaben Termin 4}
    \begin{quote}
        In dem zweiten Teil der Vorbereitungsaufgaben ging es um Fensterung. Mit
        unterschiedlichen Fensterfunktionen kann man Analysefenster erschaffen,
        die ein ganzzahliges Vielfaches der Periodendauer des Messsignals lang
        sind. Somit kann man bei der Untersuchung der Signale Leck-Effekte
        verhindern.\\
       
        \subsubsection{Matlabfunktion-Spektrum}
		\begin{quote}
            Dafür machten wir uns mit unterschiedlichen Fensterfunktionen vertraut und schrieben eine MATLAB Funktion,
            mit der ein Fenster generiert werden konnte, dessen Länge genau einer Periodenlänge des verwendeten Signals
            entspricht. Zusätzlich wurden Fenster und Signal überlagert und die DFT gebildet, Betrags- und
            Phasenspektren wurden geplottet.\\
            Das dazugehärige Matlabscripte Spektrum.m sowie Vorbereitungsaufgabe42.m befindet sich im Anhang.\\
            In diesem Skript haben wir ein Sinussignal mit der Frequenz $100$Hz erzeugt und jeweils mit einem Rechteck-,
            einem BlackMan-, einem Hanning- sowie einem verkürztem Hanningfenster multipliziert.\\
            Die Plotts dazu sind hier:
            
            \begin{figure}[H]
            \centering
                \includegraphics[scale=0.5, trim = 1.5cm 7cm 1.5cm 8cm, clip]{./Bilder/Rechteckwindow}
                    \caption{Rechteckwindow}
            \end{figure}
            
            \begin{figure}[H]
            \centering
                \includegraphics[scale=0.5, trim = 1.5cm 7cm 1.5cm 8cm, clip]{./Bilder/BlackManwindow}
                    \caption{BlackManwindow}
            \end{figure}
    
            \begin{figure}[H]
            \centering
                \includegraphics[scale=0.5, trim = 1.5cm 7cm 1.5cm 8cm, clip]{./Bilder/Hanningwindow}
                    \caption{Hanningwindow}
            \end{figure}
            
            \begin{figure}[H]
            \centering
                \includegraphics[scale=0.5, trim = 1.5cm 7cm 1.5cm 8cm, clip]{./Bilder/Hanningwindowverkuerzt}
                    \caption{Hanningwindow verkürzt}
            \end{figure}
            
            Die Letzte Simulation zeigt eine Fensterung, die zu dem Leck-Effekt führt. Als Ergebniss ist ein beiteres
            und flacheres Spektrum zu sehen, wie es beim Leckeffekt zu erwarten war.
    			
		\end{quote}

        \subsubsection{DFT eines Rechteck- und Hanningfenster}
		\begin{quote}
            Außerdem untersuchten wir die Spektren von einem Hanning- und einem Rechteckfenster, zuerst mit einer
            Fensterlänge von $16$ und danach mit einer Länge von $2^20$, wobei die restlichen Werte durch Null-Padding
            eingefügt wurden.\\
            Das dazugehörige Skript Vorbereitungsaufgabe43.m befindet sich im Anhang
            Die Plotts dazu sind hier:
            
                %4 Grafiken:
                \begin{center}
                \begin{tabular}{ll}
    
                \hspace{-11em}
                    \begin{minipage}{0.6\textwidth}
    
                        \begin{figure}[H]
                            \label{fig:}
                            \includegraphics[scale=0.5, trim = 1.5cm 7cm 1.5cm 8cm, clip]{./Bilder/RechteckDFT}
                            %FIXME [width=640px,
                             %height=474px]
                            \caption{DFT eines Rechteckfensters}
                        \end{figure}
    
                    \end{minipage}
                    \begin{minipage}{0.6\textwidth}
    
                        \begin{figure}[H]
                            \label{fig:}
                            \includegraphics[scale=0.5, trim = 1.5cm 7cm 1.5cm 8cm, clip]{./Bilder/HanningDFT}
                            %FIXME [width=640px,
                             %height=474px]
                            \caption{DFT eines Hanningfensters}
                        \end{figure}
                    \vspace{-1.5em}
    
                    \end{minipage}
    
                \end{tabular}
                \end{center}
    
                            %4 Grafiken:
                \begin{center}
                \begin{tabular}{ll}
    
                \hspace{-11em}
                    \begin{minipage}{0.6\textwidth}
    
                        \begin{figure}[H]
                            \label{fig:}
                            \includegraphics[scale=0.5, trim = 1.5cm 7cm 1.5cm 8cm, clip]{./Bilder/RechteckDFTzeropadding} %FIXME [width=640px,
                            % height=474px]
                            \caption{DFT eines Rechteckfensters mit Zeropadding}
                        \end{figure}
    
                    \end{minipage}
                    \begin{minipage}{0.6\textwidth}
    
                         \begin{figure}[H]
                            \label{fig:}
                            \includegraphics[scale=0.5, trim = 1.5cm 7cm 1.5cm 8cm, clip]{./Bilder/HanningDFTzeropadding} %FIXME [width=640px,
                            % height=474px]
                            \caption{DFT eines Hanningfensters mit Zeropadding}
                        \end{figure}
                   \vspace{-1.5em}
    
                    \end{minipage}
    
                \end{tabular}
                \end{center}
                
        Wir können sehen, dass die DFT-Spektren der Fenster mit dem Zero-Padding
        den erwarteten Sifunkionen entsprechen, während die DFT-Spektren der
        Fenster mit der Länge von $16$ es nicht tun. Dieses Ergebnis liegt
        daran, dass durch die DFT das Zeitsignal periodisch erweitert wird. Wenn
        beispielsweise beim Rechteck ein Signal mit der konstanten Amplitude von
        $1$ unendlich erweitert wird, erhält man ein unendlich ausgedehntes Rechteck mit der
        Amplitude $1$. Bei der DFT wird das Signal dann nicht als das
        Ursprungssignal erkannt und die Spektren weichen von der Erwartung ab.
        Wenn aber durch das Zero-Padding das Rechteck in der unendlichen Periodizität hervorgehoben wird, 
        kann das Signal als Rechteck abgetastet und in eine korrekte Sifunktion transformiert
        werden. Das gleiche Ereignis macht sich auch beim Hanningfenster
        sichtbar. 
            
		\end{quote}
    \end{quote}
\end{quote}

%--------------------------------------------------------------------
%--------------------------------------------------------------------

\section{Termin 3: Spektralanalyse I - DFT}
\begin{quote}

	\subsection{Schaltungsaufbau}
	\begin{quote}
	Im praktischen Teil des dritten Termins wurden nun reelle Messwerte aufgenommen
	und anhand der Diskreten Fouriertransformation untersucht.\\
	Dazu wurde zunächst eine Schaltung aufgebaut, in der der Strom im
	Dimmerschaltkreis erfasst werden konnte. Der Strom aus der Steckdose führt
	dabei in die Dimmerschaltung und dann in den Stromwandler der Wandler-Box. Dort
	wird das Signal in eine Spannung umgewandelt. Das Verhältnis von Strom zu
	Spannung ist bei dem Wandler linear. Bei einem Maximalstrom von $0.9 A$ kann
	man eine Maximalspannung von $\pm 10 V$ erhalten. Daher muss der Vorfaktor von
	$\frac{0.9 A}{10 V} = 0.09 \frac{A}{V}$ berücksichtigt werden. Noch in der
	Wandler-Box wird dieses Signal gefiltert und führt zum Sensorknoten, wo anhand
	der Spannungsmessung Rückschlüsse auf den Strom gemacht werden können.
	
	\begin{figure}[H]
    \centering
        \includegraphics[scale=0.7, trim = 0cm 0cm 0cm 0cm, clip]{./Bilder/Schaltungwandlerbox}
            \caption{Aufbau Wandlerbox}
    \end{figure}
    
	\cite{Schaltungwandlerbox}
	\end{quote}
	
	\subsection{MATLAB-Skript für Anschnittswinkel-Bestimmung}
	\begin{quote}	
	Als nächstes soll anhand eines selbstgeschriebenem MATLAB-Skripts der
	Anschnittswinkel der unterschiedlichen Signale bestimmt werden. Die
	Anschnittswinkel werden dafür vorher mit dem Oszilloskop eingestellt. Da es
	manuell nicht so einfach ist, wurden kleine Abweichungen toleriert.
	
	Unsere anhand des MATLAB-Skripts errechneten Werte wichen nur minimal von den
	erwarteten Anschnittswinkeln ab. Das Ergebnis war also zufriedenstellend. 
    Das dazugehörige Skript ist am Ende des Protokolls zu sehen.
   	\end{quote}
	
	\subsection{Amplituden- und Phasenspektrum des Stroms anhand der DFT}
	\begin{quote}
        In diesem Teil bestand die Aufgabe darin für einen selbst gewählten Anschnittswinkel den Stromverlauf
        aufzunehmen und das Amplituden sowie das Phasenspektrum aufzunehmen. Es sollte eine passende Abtastfrequenz und
        ein passendes Messfenster gewählt werden. Außerdem sollte der Einfluss des Antialiasing-Filters auf den
        Betragsfrequenzgang korrigiert werden.\\
        Für unsere Messung haben wir uns entschieden mit einer Abtastfrequenz von $15kHz$ und einem Messfenster von
        $600$ Messwerten abzutasten. Die Abtastfrequenz haben wir gewält, da sie fast die maximal mögliche
        Abtastfrequenz des Messknotens ist und wir mit mehr Abtastwerten eine größere Genauigkeit erreichen. Als
        Zeitfenster haben wir zwei Perioden gewählt, was bei einer Frequenz von $50Hz$ eine Zeitfenster von $0$ bis
        $0.04 s$ bedeutet. Bei der gewählten Abtastfrequenz ergibt sich daraus ein Messfenster von $600$
        Messwerten.\vspace{1em}
        
        Um die Auswirkung des Antialiasing-Filters auf den Betragsfrequenzgang zu korrigieren betrachen wir zunächst den
        Betragsfrequenzgang des Filters selbst den wir in einem früheren Termin aufgenommen haben. Wie gewollt dämpft
        das Filter alle Frequenzen größer als $5kHz$ mit mehr als $60dB$ und ist damit nicht mehr von dem ADU messbar.
        Um dieses Ziel zu erreichen beginnt das Filter schon ab einer Frequenz von $1,2 kHz$ zu dämfen und verfälscht
        dadurch die Amplituden der Messwerte der Frequenzen zwischen $1,2 kHz$ und $5 kHz$.\vspace{1em}
        
        Diesen Fehler korrigieren wir indem wir den Kehrwert des Amplitudengangs des Filters mit dem Betragsspektrum der
        Messwerte multiplizieren. Dafür benötigen wir genausoviele Abtastwerte des Amplitudengangs des Filters wie wir
        auch für die gesamte Messung gewählt haben: $600$. Bei der manuellen Messung des Frequenzganges haben wir jedoch
        lediglich $18$ Messwerte aufgenommen. Unser erster Gedanke war eine Interpolation des gemessenen Frequenzgang zu
        erstellen und daraus einen passenden Vektor zu erstellen. Leider hatten wir dabei einige
        Umsetzungsschwierigkeiten wesshalb uns unser Tutor den Tipp gab anstatt des reell gemessenen Filters das
        simulierten Filter zu benutzen.\\
        Daher haben wir uns mit Hilfe des bode Befehls den Betragsgang des simulierten Butterworth Tiefpass 8.Ordnung
        für die selben Frequenzen ausgeben lassen mit denen auch abgetastet wurde. Jedoch mussten wir noch beachten,
        dass das abgetastete Signal bei der DFT doppelt vorkommt und auch den Betragsgang daran anpassen müssen. Da wir
        uns dafür entschieden haben in $dB$ zu arbeiten haben wir diesen Betragsgang von $1$ abgezogen um den
        invertierten Betragsfrequenzgang zu bekommen.\vspace{1em}
        
        Zur Korrektur haben wir diesen invertierten Betragsfrequenzgang bei der betrechnung des Spektrum des gemessenen
        Signals an das errechnete Betragsspektrum addiert. Anschließend haben wir das korrigierte Spektrum geplottet.
        Außerdem haben wir den korrigierten und den unkorrigierten Betragsfrequenzgang übereinandergelegt.\\
        Hier sind die Ergebnisse:\\
        
        \begin{figure}[H]
        \centering
            \includegraphics[scale=0.7, trim = 1.5cm 7cm 1.5cm 8.5cm, clip]{./Bilder/korrigiertesSpektrumauf33}
                \caption{korregiertes Spektrum des aufgenommenen Stromverlaufs}
        \end{figure}

        \begin{figure}[H]
        \centering
            \includegraphics[scale=0.7, trim = 1.5cm 7cm 1.5cm 8.5cm, clip]{./Bilder/betragsfrequenzgang_korr_vs_unkorr.pdf}
                \caption{Korregierter und unkorrigierter Betrgsfrequenzgang}
        \end{figure}
        
        Wie zu erwarten war hat die Korrektur auf den ersten Blick keinen auffallenden einfluss. Das Betragspektrum
        zeigt, dass die Frequenzen, die durch das Antialiasing Filter beeinflusst werden sowieso schon eine kleine
        Amplitude besitzen. Um jedoch die Auswirkungen der Korrektur genauer zubetrachten Vergrößern wir den
        Betragsfrequenzgang auf den interessanten Bereich der hohen Frequenzen.
        
        \begin{figure}[H]
        \centering
            \includegraphics[scale=0.7, trim = 1.5cm 7cm 1.5cm 8.5cm,
            clip]{./Bilder/betragsfrequenzgang_korr_vs_unkorr_zoom}
                \caption{Korregierter und unkorrigierter Betragsfrequenzgang}
        \end{figure}
    
        In dieser Grafik lässt sich gut erkennen, dass die Korrektur die Amplituden der hohen Frequenzen verstärkt die
        Amplituden der niedrigen Frequenzen jedoch nahezu unverändert lässt. Ein solches Verhalten haben wir erwartet.
	\end{quote}
	
	\subsection{Effektivwert des Stroms im Zeit- und Frequenzbereich}
	\begin{quote}
	Zuletzt sollte der Effektivwert des Stroms einer Messung im Zeit- und
	Frequenzbereich ermittelt werden, indem der MATLAB-Skript der
	Vorbereitungsaufgabe verwendet wurde. Wir verwendeten den Strom des Signals bei
	einem Anschnittswinkel von $\frac{\pi}{2}$.\\
	Das MATLAB-Skript gibt uns das Ergebnis von $0.21 A$ für Zeit- und
	Frequenzbereich. Der mithilfe des Multimeters gemessene Strom beträgt $0.224
	A$.\\
	
	Vergleicht man beide Messungen, kann man sehen, dass es keine erhebliche
	Differenz gibt.
	
	 
	\end{quote}
\end{quote}

\section{Termin 4: Spektralanalyse II - Fensterung}
\begin{quote}

	Im Labor soll nun ein phasenangeschnittener Strom gemessen und mit der DFT untersucht werden.
	Dazu wird der Dimmer auf dem "Lampenbrett" verwendet.
	
	\subsection{Versuchsaufbau}
	\begin{quote}
	Allgemein gilt: Um Strom zu messen muss man ein Amperemeter in Reihe mit der Last schalten. 
	Daher muss auch hier der Stromwandler aus der "blauen Box" in Reihe zur Last geschaltet 
	werden. Das in eine Spannung gewandelte und gefilterte Signal wird mit dem Sensorknoten gemessen. 
	Das Messsignal wird anschließend mit Matlab weiterverarbeitet.

	\begin{figure}[htb]
	\centering
	\includegraphics[scale=0.7, width=1\textwidth]{./Bilder/Versuchsaufbau1}
	\caption{Versuchsaubau um angeschnittenen Strom zu messen}
	\end{figure}
	
	\end{quote}
		
	\subsection{Versuchsdurchführung}
	\begin{quote}
		
		\subsubsection{Phasenanschnitt messen}
		\begin{quote}
		An dem Phasenanschnittdimmer wird ein Zündwinkel eingestellt. Dieser kann aber nicht am Dimmer 
		bestimmt werden, da eine Skala fehlt. Statt dessen wird das Phasenangeschnittene Signal zunächst 
		mit dem Oszilloskop gemessen. Dabei wird die Zeitdifferenz zwischen dem
		Nulldurchgang und Zündmoment gemessen. Anschließend wird das Signal auch mit dem Sensorknoten abgetastet.
		Es werden Messpunkte bei vielfachen von ca. $2,5ms$ Phasenverschub
		aufgenommen.
		Anschließend wird ein Matlabskript geschrieben, dass den Zündwinkel aus den Messwerten bestimmt.\\ 
		Um den Leckeffekt von vornerein zu umgehen, wird als Messdauer ein ganzzahliges vielfaches der 
		Signalperiode ($20ms$) gewählt.\\ 
		Die Abtastfrequenz muss wie folgt gewählt werden. Die $3dB$-Grenzfrequenz des
		Allaisingfilters liegt bei $3,1kHz$, die Auflösung des ADUs beträgt $10$Bit
		und der Eingangsspannungsbereich beträgt $14$V.\\
		
		\begin{align}
		U_{LSB} = \frac{14V}{2^{10}-1} = 0,01368V 
		\end{align} 
		
		Nun kann die benötigte Dämpfung, um Allaising zu verhindern, berechnet werden.\\
		
		\begin{align}
		D = 20 \cdot log_{10}(\frac{14V}{U_{LSB}}) = 60dB
		\end{align} 
		
		Der Filter ist ein Filter $8$.Ordnung und besitzt eine Steilheit von
		$160dB$/Dekade. Es wird abgeschätzt, dass der Filter ab ca. $3,5kHz$ um über
		$60dB$ dämpft. Daher muss die Abtastfrequenz mindestens $7kHz$ betragen.
		\end{quote}
		
		\subsubsection{Auswirkungen des Fensters}
		\begin{quote}
		Es werden nun nacheinander Rechteck-, Hanning- und Blackmanfenster über das Signal gelegt. 
		Die Längen der Fenster sind absichtlich so gewählt, dass es zum Leckeffekt kommt. 
		Es wird nun untersucht, wie sich die unterschiedlichen Fenster auf den Leckeffekt auswirken. 
		\end{quote}
		
		\subsubsection{Qualität der Netzspannung}
		\begin{quote}
		Nun soll die Qualität der Netzspannung überprüft werden. Theoretisch beträgt die Netzspannung 
		$230V$ bei $50Hz$. Allerdings können auf der Grundwelle zusätzlich zu den
		$50Hz$ weitere Oberwellen vorhanden sein. Der Anteil dieser Oberwellen an der
		Versorgungsspannung darf nicht über $8\%$ liegen.\\
		Für diesen Versuch wird die Netzspannung an den Spannungswandler der Blauenbox angelegt. 
		Nun kann mit dem Sensorknoten die Netzspannung gemessen werden. Der Spannungswandler setzt die 
		Eingangsspannung um den Faktor $56,6$ herab. Um die Netzspannung zu
		bestimmen, muss der Messwert mit $56,6$ multipliziert werden.
		
		\begin{figure}[htb]
		\centering
		\includegraphics[scale=0.7, width=1\textwidth]{./Bilder/Versuchsaufbau2}
		\caption{Versuchsaubau um Netzqualität zu messen}
		\end{figure}
		
		Der Gesamtoberschwingungsgehalt (THD-Wert für Total Harmonic Distortion) wird
		anhand folgender Formel berechnet:
		
		\begin{align}
		THD = \frac{\sqrt{U_2^2 + U_3^2 + \cdot \cdot \cdot + U_N^2}}{U_1}
		\end{align} 
		
		Mit diesem Ansatz sollte ein Metzoberschwingungsanalysator auf Basis der DFT
		realisiert werden. Hierzu wurde eine Messkette entworfen.
		
		\TODO{die Ergebisse der TDH Messung fehlt hier}
		
		\end{quote}%Ende subsubsection
	\end{quote}%Ende subsection
			
\end{quote}%Ende section
%--------------------------------------------------------------------
%--------------------------------------------------------------------
\section{Quellcodes}
\begin{quote}
	
	\subsection{Codes aus Termin 3}
	\begin{quote}
	    \subsubsection{sinus2.m}
	    \begin{quote}
	        \lstinputlisting[
	            caption={sinus2},
	            label=lst:Matlab]
	            {./Matlab/sinus2.m}
	    \end{quote}
    
	    \subsubsection{stromPhasSchnitt.m}
	    \begin{quote}
	        \lstinputlisting[
	            caption={stromPhasSchnitt},
	            label=lst:Matlab]
	            {./Matlab/stromPhasSchnitt.m}
	    \end{quote}
    
	    \subsubsection{EffektivwertZeitbereich.m}
	    \begin{quote}
	        \lstinputlisting[
	            caption={EffektivwertZeitbereich},
	            label=lst:Matlab]
	            {./Matlab/EffektivwertZeitbereich.m}
	    \end{quote}
    
	    \subsubsection{EffektivwertFourier.m}
	    \begin{quote}
	        \lstinputlisting[
	            caption={EffektivwertFourier},
	            label=lst:Matlab]
	            {./Matlab/EffektivwertFourier.m}
	    \end{quote}
	    \subsubsection{alphadetektor2.m}
        \begin{quote}
            \lstinputlisting[
                caption={alphadetektor2},
                label=lst:Matlab]
                {./Matlab/alphadetektor2.m}
        \end{quote}
        
        \subsubsection{Aufgabe33.m}
        \begin{quote}
            \lstinputlisting[
                caption={Aufgabe33},
                label=lst:Matlab]
                {./Matlab/Aufgabe33.m}
        \end{quote}
        
        \subsubsection{Spektrum2Filterkorrektur.m}
        \begin{quote}
            \lstinputlisting[
                caption={Spektrum2Filterkorrektur},
                label=lst:Matlab]
                {./Matlab/Spektrum2Filterkorrektur.m}
        \end{quote}
	\end{quote}
	
	\subsection{Codes aus Termin 4}
	\begin{quote}
		
		\subsubsection{Spektrum.m}
		\begin{quote}
		\lstinputlisting[
				caption={Spektrum},
				label=lst:Spektrum]
				{./Matlab/Spektrum.m}
		\end{quote}
		
		\subsubsection{Vorbereitungsaufgabe2.m}
		\begin{quote}
		\lstinputlisting[
				caption={},
				label=lst:Matlab]
		        {./Matlab/Vorbereitungsaufgabe42.m}
		\end{quote}
		
		\subsubsection{Vorbereitungsaufgabe3.m}
		\begin{quote}
		\lstinputlisting[
				caption={Vorbereitungsaufgabe43},
				label=lst:Matlab]
                {./Matlab/Vorbereitungsaufgabe43.m}		
		\end{quote}
		
	\end{quote}
\end{quote}

%--------------------------------------------------------------------
%--------------------------------------------------------------------


\begin{thebibliography}{999}
\bibitem {Schaltungwandlerbox} Prof. Dr.-Ing. Gühmann, Clemens; Dipl.-Ing. Funk, Jürgen: MDVLaborGeraete_web, S.4

%Name, Vorname.; evtl. Name2, Vorname2.: Titel des Dokumentes
%oder Buches, Zeitschrift/Verlag/URL (Auflage, Erscheinungsort, -jahr), ggf. Seitenzahlen
\bibitem {PasevalscheTheorem} \url{https://de.wikipedia.org/wiki/Parsevalsches_Theorem}, Zugriff
23.05.2012
\end{thebibliography}


\end{document}


